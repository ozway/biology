\input{../header.tex}

\title{Work Log for November}
\author{Logan Brown}
%\date{} % Activate to display a given date or no date (if empty),
         % otherwise the current date is printed 

\begin{document}
\maketitle
\tableofcontents

\newpage


\section{Goals for the Month}
As of October 31st
\begin{enumerate}
\item Use Preston's Simulated Yeast, compare to REU yeast

look for estimated $\approx$ 4*true
\item Parallelize the Code

mclapply, getOption(``mc.cores")?
\item Wei Chen's Yeast / Real Yeast Genome
\item Generate my own simulated yeast, using a reverse engineered cubfits
\end{enumerate}

%Paste output from writeGoals
\section{Progress/Notes}

\subsection{Use Preston's Simulated Yeast, compare to REU yeast}

Here are the results from Preston's simulated yeast...

Notice that the $\Delta\omega$ values are not off by that $\approx4$ factor. We got a pretty good correlation on the log$\mu$ values, but the $\phi$ is pretty lousy, and the $\omega$ values leave something to be desired.

\includepdf[pages={1}]{data/1105_codonParameters_8core.pdf}
\includepdf[pages={1}]{data/1105_CUB_est_bin_8core.pdf}
\includepdf[pages={1}]{data/1105_CUB_obs_bin_8core.pdf}
\includepdf[pages={1}]{data/1105_deltaeta_8core.pdf}
\includepdf[pages={1}]{data/1105_expPhi_trace_8core.pdf}
\includepdf[pages={1}]{data/1105_histogram_8core.pdf}
\includepdf[pages={1}]{data/1105_logmu_8core.pdf}
\includepdf[pages={1}]{data/1105_pMat_trace_8core.pdf}
\includepdf[pages={1}]{data/1105_vs_obs_phi_8core.pdf}

Note how different the $\Delta\omega$ values are between Preston's yeast genome and the REU students' yeast genome.

\includepdf[pages={1}]{data/1105_sidebyside.pdf}


\subsection{Visualization}

\subsubsection{'true' values vs simulated values}

Changes have been added to visualize.r, based on other plotting functions.

I've added confidence intervals (the scale of the interval can bet set in visualize.r). Cedric didn't have any functions to do so, but I was able to apply the "plotrix" package. I've also installed that package to "/home/lbrown/cubfits/Dependencies/plotrix". Everyone else should have permissions on that directory, in case someone wants to use my edited visualize.r function.




\subsection{Parallelize the Code}

\subsubsection{getOption(``mc.cores")}
How to set the number of cores for an mclapply call? mclapply's default number of cores is getOption(``mc.cores",2L);

getOption(``(option)", (value)) returns the value previously set to that (option), or otherwise it returns (value). mc.cores is not set by default. So first, set option(``mc.cores"=Number\_of\_Cores). Then mclapply should correctly get the number of cores.

\subsubsection{Timing}

As expected, we get diminshing returns on adding additional processors

\includegraphics[width=0.5\textwidth]{data/1105runtimes.png}
\includegraphics[width=0.5\textwidth]{data/1105timingCostBenefit.png}


\subsection{Add (a1-a2) as a parameter}

In the code, when the posterior probability of a codon is calculated, instead of calculating

\[
\mbox{Pr}(c_i|\phi,i)
=
\frac{
\mbox{exp}[\mbox{ln}
+ \omega_i(a_1-a_2)y_1
+ \omega_ia_2y_1 i
]
}{
\mbox{exp}[\mbox{ln}
\sum_{u=1}^m
\mbox{exp}[\mbox{ln}
+ \omega_i(a_1-a_2)y_1
+ \omega_ia_2y_1 i
]
}
\]

Wei Chen calculates

\[
\mbox{Pr}(c_i|\phi,i)
=
\frac{
\mbox{exp}[\mbox{ln}
%+ \omega_i(a_1-a_2)y_1
- \omega_i\phi i
]
}{
\mbox{exp}[\mbox{ln}
\sum_{u=1}^m
\mbox{exp}[\mbox{ln}
%+ \omega_i(a_1-a_2)y_1
- \omega_i\phi i
]
}
\]

This was done for a number of reasons. The $y_1$ term is just the aggregate of the effective population, $\phi$, and a scaling term $-q$. Also, the assumption was that $a_1 \approx a_2 = 4$ATP.
To better account for the parameters of the model, we're going to add another parameter called $\Delta a_{12} = (a_1 - a_2)$, and use

\[
\mbox{Pr}(c_i|\phi,i)
=
\frac{
\mbox{exp}[\mbox{ln}
- \omega_i(\Delta a_{12})\phi
- 4\omega_i\phi i
]
}{
\mbox{exp}[\mbox{ln}
\sum_{u=1}^m
\mbox{exp}[\mbox{ln}
- \omega_i(\Delta a_{12})\phi
- 4\omega_i\phi i
]
}
\]

The math part of the change takes place in my.logdmultinomCodOne.r, adding an extra row to xm and multiplying $\omega\phi\Delta a_{12}$, which is baa[2]*tmp.phi*(new parameter)

\subsection{Wei Chen's Yeast / Real Yeast Genome}

\subsection{Generate my own simulated yeast, using a reverse engineered cubfits}



\section{Goals for next Month}
\begin{enumerate}
\item Future Goal
\end{enumerate}


\end{document} %End of day document, REMOVE
