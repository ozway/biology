% !TEX TS-program = pdflatex
% !TEX encoding = UTF-8 Unicode

% This is a simple template for a LaTeX document using the "article" class.
% See "book", "report", "letter" for other types of document.

\documentclass[11pt]{article} % use larger type; default would be 10pt


\usepackage{ulem}
\newcommand\NoIndent[1]{%
  \par\vbox{\parbox[t]{\linewidth}{#1}}%
}


\usepackage[utf8]{inputenc} % set input encoding (not needed with XeLaTeX)

%%% Examples of Article customizations
% These packages are optional, depending whether you want the features they provide.
% See the LaTeX Companion or other references for full information.

%%% PAGE DIMENSIONS
\usepackage{geometry} % to change the page dimensions
\geometry{a4paper} % or letterpaper (US) or a5paper or....
% \geometry{margin=2in} % for example, change the margins to 2 inches all round
% \geometry{landscape} % set up the page for landscape
%   read geometry.pdf for detailed page layout information

\usepackage{graphicx} % support the \includegraphics command and options

% \usepackage[parfill]{parskip} % Activate to begin paragraphs with an empty line rather than an indent

%%% PACKAGES
\usepackage{booktabs} % for much better looking tables
\usepackage{array} % for better arrays (eg matrices) in maths
\usepackage{paralist} % very flexible & customisable lists (eg. enumerate/itemize, etc.)
\usepackage{verbatim} % adds environment for commenting out blocks of text & for better verbatim
\usepackage{subfig} % make it possible to include more than one captioned figure/table in a single float
% These packages are all incorporated in the memoir class to one degree or another...

%%% HEADERS & FOOTERS
\usepackage{fancyhdr} % This should be set AFTER setting up the page geometry
\pagestyle{fancy} % options: empty , plain , fancy
\renewcommand{\headrulewidth}{0pt} % customise the layout...
\lhead{}\chead{}\rhead{}
\lfoot{}\cfoot{\thepage}\rfoot{}

%%% SECTION TITLE APPEARANCE
\usepackage{sectsty}
\allsectionsfont{\sffamily\mdseries\upshape} % (See the fntguide.pdf for font help)
% (This matches ConTeXt defaults)

%%% ToC (table of contents) APPEARANCE
\usepackage[nottoc,notlof,notlot]{tocbibind} % Put the bibliography in the ToC
\usepackage[titles,subfigure]{tocloft} % Alter the style of the Table of Contents
\renewcommand{\cftsecfont}{\rmfamily\mdseries\upshape}
\renewcommand{\cftsecpagefont}{\rmfamily\mdseries\upshape} % No bold!

%%% END Article customizations


\usepackage{verbatim}
\usepackage{amsmath}


\title{Work Log for January}
\author{Logan Brown}
%\date{} % Activate to display a given date or no date (if empty),
         % otherwise the current date is printed 

\begin{document}
\maketitle
\tableofcontents

\newpage


\section{Goals for the Month}
%Paste output from writeGoals
\begin{enumerate}
\item Generate new genomes
\item Speed up NSE model
\item Move to Newton

\end{enumerate}

\section{Progress/Notes}

\subsection{Generate new genomes}

1/5: Fixed a small bug in the genome code.

Now that the code actually updates the codons, not just the codon index, we can generate new genomes.

yeast2.fasta is just redone Preston's yeast. moddedYeast.fasta uses the nse probabilities from CUBFits.


\subsection{Speed up NSE model}

\subsubsection{Move log into C?}

Right now, the code exponentiates the values in the C code (to be normalized), then later, in the R code, calulates the logarithm of those values, to correct this. Since C code is generally faster than R code, I thought it may be worthwhile to move that calculation into the C code.

I ran a quick test case (the code can be found in data/cLogTest.tar.gz), and it seems like making that change would save about 10 nanoseconds per gene. Our simulated yeast has 2.8 million genes, so it's only about a .3 second improvement per MCMC proposal. Basically, this change is too minor.

\subsection{Move to Newton}





\section{Goals for next Month}
\begin{enumerate}
\item Future Goal
\end{enumerate}


\end{document} %End of day document, REMOVE
