% !TEX TS-program = pdflatex
% !TEX encoding = UTF-8 Unicode

% This is a simple template for a LaTeX document using the "article" class.
% See "book", "report", "letter" for other types of document.

\documentclass[11pt]{article} % use larger type; default would be 10pt


\usepackage{ulem}
\newcommand\NoIndent[1]{%
  \par\vbox{\parbox[t]{\linewidth}{#1}}%
}


\usepackage[utf8]{inputenc} % set input encoding (not needed with XeLaTeX)

%%% Examples of Article customizations
% These packages are optional, depending whether you want the features they provide.
% See the LaTeX Companion or other references for full information.

%%% PAGE DIMENSIONS
\usepackage{geometry} % to change the page dimensions
\geometry{a4paper} % or letterpaper (US) or a5paper or....
% \geometry{margin=2in} % for example, change the margins to 2 inches all round
% \geometry{landscape} % set up the page for landscape
%   read geometry.pdf for detailed page layout information

\usepackage{graphicx} % support the \includegraphics command and options

% \usepackage[parfill]{parskip} % Activate to begin paragraphs with an empty line rather than an indent

%%% PACKAGES
\usepackage{booktabs} % for much better looking tables
\usepackage{array} % for better arrays (eg matrices) in maths
\usepackage{paralist} % very flexible & customisable lists (eg. enumerate/itemize, etc.)
\usepackage{verbatim} % adds environment for commenting out blocks of text & for better verbatim
\usepackage{subfig} % make it possible to include more than one captioned figure/table in a single float
% These packages are all incorporated in the memoir class to one degree or another...

%%% HEADERS & FOOTERS
\usepackage{fancyhdr} % This should be set AFTER setting up the page geometry
\pagestyle{fancy} % options: empty , plain , fancy
\renewcommand{\headrulewidth}{0pt} % customise the layout...
\lhead{}\chead{}\rhead{}
\lfoot{}\cfoot{\thepage}\rfoot{}

%%% SECTION TITLE APPEARANCE
\usepackage{sectsty}
\allsectionsfont{\sffamily\mdseries\upshape} % (See the fntguide.pdf for font help)
% (This matches ConTeXt defaults)

%%% ToC (table of contents) APPEARANCE
\usepackage[nottoc,notlof,notlot]{tocbibind} % Put the bibliography in the ToC
\usepackage[titles,subfigure]{tocloft} % Alter the style of the Table of Contents
\renewcommand{\cftsecfont}{\rmfamily\mdseries\upshape}
\renewcommand{\cftsecpagefont}{\rmfamily\mdseries\upshape} % No bold!

%%% END Article customizations


\usepackage{verbatim}


\begin{document}

%%% To switch from 1 + p/(1-p) to 1/(1-p), try...
%s/1+\\frac{p_{j}}{1-p_{j}}/\\\frac{1}{1-p_{j}}/g

%%% To switch to 1 + p/(1-p) from 1/(1-p), try...
%s/1+\\frac{p_{j}}{1-p_{j}}/1+\\frac{p_{j}}{1-p_{j}}/g

Calculation of the cost over the benefit of codon $c$ at position $i$ where\\
$a_1$ is the static cost of genome building\\
$a_2$ is the general cost of reading a codon (assumed constant, likely not)\\
$n$ is the length of the genome\\
$p_{ic}$ is the probability of a nonsense error at position $i$ using codon $c$\\
$p_j$ is the probability of a nonsense error at position $j$ (not knowing the codon?)\\


At position $i$...

$$\frac{\mbox{Expected Cost}}{\mbox{Expected Benefit}} =
\frac{[\mbox{(probability we reach $i$)(cost of reaching $i$)(probability of failure)}}
{ \sum_{i=1}^{n}(\mbox{value}_i)(\mbox{probability}_i) }$$

Since $\mbox{value}_i = 0$ for all $i<n$,

$$\frac{\mbox{Expected Cost}}{\mbox{Expected Benefit}} =
\frac{[\mbox{(probability we reach $i$)(cost of reaching $i$)(probability of failure)}}
{ \mbox{probability we reach $n$}}$$


$$=
\frac{
\left[\prod_{j=1}^{i-1}(1 - p_{j})\right]
\left[\sum_{k=1}^{i} a_1 + a_2(k-1)\right]
\left[p_{ic}\right]
}{
\prod_{j=1}^{n}(1 - p_{j})
}$$


$$=
\left[\prod_{j=i}^{n}\frac{1}{1 - p_{j}}\right]
\left[\sum_{k=1}^{i} a_1 + a_2(k-1)\right]
\left[p_{ic}\right]
$$


$$=
\left[\prod_{j=i+1}^{n}\frac{1}{1 - p_{j}}\right]
\left[\sum_{k=1}^{i} a_1 + a_2(k-1)\right]
\left[\frac{p_{ic}}{1-p{ic}}\right]
$$


%%$$f(p_{ic})=
%%\left[\prod_{j=i+1}^{n}(\frac{1}{1-p_{j}})\right]
%%\left[\sum_{k=1}^{i} a_1 + a_2(k-1)\right]
%%\left[\frac{p_{ic}}{1-p_{ic}}\right]
%%$$

\end{document}
