\documentclass[11pt]{article} % use larger type; default would be 10pt
\usepackage{ulem}
\newcommand\NoIndent[1]{%
  \par\vbox{\parbox[t]{\linewidth}{#1}}%
}
\usepackage[utf8]{inputenc} % set input encoding (not needed with XeLaTeX)
\usepackage{geometry} % to change the page dimensions
\geometry{a4paper} % or letterpaper (US) or a5paper or....
\usepackage{graphicx} % support the \includegraphics command and options
\usepackage{booktabs} % for much better looking tables
\usepackage{array} % for better arrays (eg matrices) in maths
\usepackage{paralist} % very flexible & customisable lists (eg. enumerate/itemize, etc.)
\usepackage{verbatim} % adds environment for commenting out blocks of text & for better verbatim
\usepackage{subfig} % make it possible to include more than one captioned figure/table in a single float
\usepackage{fancyhdr} % This should be set AFTER setting up the page geometry
\pagestyle{fancy} % options: empty , plain , fancy
\renewcommand{\headrulewidth}{0pt} % customise the layout...
\lhead{}\chead{}\rhead{}
\lfoot{}\cfoot{\thepage}\rfoot{}
\usepackage{sectsty}
\allsectionsfont{\sffamily\mdseries\upshape} % (See the fntguide.pdf for font help)
\usepackage[nottoc,notlof,notlot]{tocbibind} % Put the bibliography in the ToC
\usepackage[titles,subfigure]{tocloft} % Alter the style of the Table of Contents
\renewcommand{\cftsecfont}{\rmfamily\mdseries\upshape}
\renewcommand{\cftsecpagefont}{\rmfamily\mdseries\upshape} % No bold!
\usepackage{verbatim}
\usepackage{amsmath}
\usepackage{amssymb}

%\DeclareMathSizes{10}{10}{10}{10}

%
% The probability of seeing a whole genome
%

\begin{document}


\section{Mike Questions}
\begin{enumerate}
\item Is there a difference between $\vec{c_i}$ and $c_i$? Should we just use $\vec{c}$ and $c_i$?
\item Why does a $\vec{c_i}$ term appear in the fitness? is $\omega_{c_i}$ a function of $\vec{c_i}$ in this notation?
\item $\vec{c}$ denotes the current gene, correct? Or does it represent the entire genome?
\item If $\vec{c}$ does denote the whole genome, should $\phi$ have some sort of subscript relating it to the current gene in the genome?
\item Where do the mutation terms go? I think I'll include them for completeness. Just turn
$\mbox{exp}\left[-\sum_{j} \beta_{c_j} \omega_{c_j} (\vec{c}_j) \phi \right]$
into
$\mbox{exp}\left[-\sum_{j} (\mu_{c_j} - \beta_{c_j} \omega_{c_j} (\vec{c}_j)) \phi \right]$?
\end{enumerate}


\section{Notation}

$\vec{c}$ is the vector of codons (of length $n$, the length of the gene)

$\vec{c}_i$ or $c_i$ is the $i^{th}$ codon of $\vec{c}$

$\vec{c}_{i\ell}$ is the $\ell^{th}$ synonym of codon $c_i$

$\vec{\theta}$ is our given conditions, including $\beta, \phi, \omega$, and the effective population size

$\sum_j$ is shorthand for $\sum_{j=1}^n$, the sum over the whole gene

$\beta_\chi$ is the cost of codon translation for codon $\chi$. In the NSE model, this is constant, $\beta_1 = \beta_2 = \cdots = \beta_n$. In the ROC model, this is the main variable.

$\omega_\chi$ is the odds of a nonsense error for codon $\chi$, $\frac{p_\chi}{1-p_\chi}$ 

$\phi$ is the expression level of the gene

$\mathbb{C}$ is the set of all possible synonymous sequences

$\nu$ is the number of synonymous codons to $\vec{c}_i$

\section{Calculation}

``The probability of seeing codon $\vec{c}_i$ at position $i$ (given our conditions) is approximately the fitness of that sequence divided by the sum of the fitness of all possible synonymous sequences"
\[
P(\vec{c}_i | \vec{\theta})
\approx
\frac{
\mbox{exp}\left[-\sum_{j} \beta_{c_j} \omega_{c_j} (\vec{c}_j) \phi \right]
}{
\sum _{K \in \mathbb{C}}
\mbox{exp}\left[-\sum_{j} \beta_{K_j} \omega_{K_j} (\vec{c}_j) \phi \right]
}
\]


%\noindent{For simplicity, I'm going to rewrite exp$\left[-\sum_{j} \beta_{K_j} \omega_{K_j} (\vec{c}_j) \phi \right]$ as exp[$fitness$]
Rewrite the sum as the sum across the synonyms.
Added notation: 

$\nu$ is the number of synonyms for the given codon $i$

\[
=
\frac{
\sum _{\kappa \in \mathbb{C} | \kappa_i = c_i}
\mbox{exp}\left[-\sum_{j} \beta_{\kappa_j} \omega_{\kappa_j} (\vec{\kappa}_j) \phi \right]
}{
\sum_{\ell=1}^\nu
\sum _{K \in \mathbb{C} | K_{i\ell} = c_{i\ell}}
\mbox{exp}\left[-\sum_{j} \beta_{K_j} \omega_{K_j} (\vec{c}_j) \phi \right]
}
\]

Pull out the term for the current codon in the chain, where $c_j=c_i$

\[
=
\frac{
\left[
\sum _{\kappa \in \mathbb{C}}
\mbox{exp}\left[-\sum_{j\neq i} \beta_{\kappa_j} \omega_{\kappa_j} (\vec{\kappa}_j) \phi \right]
\right]
\left(\mbox{exp}\left[-\beta_{c_i} \omega_{c_i} (\vec{c}_i) \phi \right]\right)
}{
\sum_{\ell=1}^\nu
\sum _{K \in \mathbb{C} | K_{i\ell} = c_{i\ell}}
\mbox{exp}\left[-\sum_{j} \beta_{K_j} \omega_{K_j} (\vec{K}_j) \phi \right]
}
\]


Additional expansion that wasn't on the board. Pull out the $K_{i\ell}$ term, knowing that $K_{i\ell}=c_{i\ell}$

\[
=
\frac{
\left[
\sum _{\kappa \in \mathbb{C}}
\mbox{exp}\left[-\sum_{j\neq i} \beta_{\kappa_j} \omega_{\kappa_j} (\vec{\kappa}_j) \phi \right]
\right]
\left(\mbox{exp}\left[-\beta_{c_i} \omega_{c_i} (\vec{c}_i) \phi \right]\right)
}{
\sum_{\ell=1}^\nu
\sum _{K \in \mathbb{C} | K_{i\ell} = c_{i\ell}}
\mbox{exp}\left[-\sum_{j\neq i} \beta_{K_j} \omega_{K_j} (\vec{K}_j) \phi \right]
\mbox{exp}\left[-\beta_{c_{i\ell}} \omega_{c_{i\ell}} (\vec{c}_{i\ell}) \phi \right]
}
\]

The $\ell$ term in the denominator can be pulled out of the inner sum, leaving

\[
=
\frac{
\left[
\sum _{\kappa \in \mathbb{C}}
\mbox{exp}\left[-\sum_{j\neq i} \beta_{\kappa_j} \omega_{\kappa_j} (\vec{\kappa}_j) \phi \right]
\right]
\left(\mbox{exp}\left[-\beta_{c_i} \omega_{c_i} (\vec{c}_i) \phi \right]\right)
}{
\sum_{\ell=1}^\nu
\left(
\mbox{exp}\left[-\beta_{c_{i\ell}} \omega_{c_{i\ell}} (\vec{K}_{i\ell}) \phi \right]
\sum _{K \in \mathbb{C} | K_{i\ell} = c_{i\ell}}
\mbox{exp}\left[-\sum_{j\neq i} \beta_{K_j} \omega_{K_j} (\vec{K}_j) \phi \right]
\right)
}
\]

Since the inner sum no longer relies on $\ell$, we can restate the above as 

\[
=
\frac{
\left[
\sum _{\kappa \in \mathbb{C}}
\mbox{exp}\left[-\sum_{j\neq i} \beta_{\kappa_j} \omega_{\kappa_j} (\vec{\kappa}_j) \phi \right]
\right]
\left(\mbox{exp}\left[-\beta_{c_i} \omega_{c_i} (\vec{c}_i) \phi \right]\right)
}{
\sum_{\ell=1}^\nu
\left(
\mbox{exp}\left[-\beta_{c_{i\ell}} \omega_{c_{i\ell}} (\vec{C}_{i\ell}) \phi \right]
\right)\left(
\sum _{K \in \mathbb{C}}
\mbox{exp}\left[-\sum_{j\neq i} \beta_{K_j} \omega_{K_j} (\vec{K}_j) \phi \right]
\right)
}
\]


Which cancels to


\[
P(\vec{c}_i | \vec{\theta})
\approx
\frac{
\mbox{exp}\left[-\beta_{c_i} \omega_{c_i} (\vec{c}_i) \phi \right]
}{
\sum_{\ell=1}^\nu
\mbox{exp}\left[-\beta_{c_{i\ell}} \omega_{c_{i\ell}} (\vec{c}_{i\ell}) \phi \right]
}
\]

%\input{genomeProb.logan.tex}

\end{document}
