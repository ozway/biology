\documentclass[11pt]{article} % use larger type; default would be 10pt
\usepackage{ulem}
\newcommand\NoIndent[1]{%
  \par\vbox{\parbox[t]{\linewidth}{#1}}%
}
\usepackage[utf8]{inputenc} % set input encoding (not needed with XeLaTeX)
\usepackage{geometry} % to change the page dimensions
\geometry{a4paper} % or letterpaper (US) or a5paper or....
\usepackage{graphicx} % support the \includegraphics command and options
\usepackage{booktabs} % for much better looking tables
\usepackage{array} % for better arrays (eg matrices) in maths
\usepackage{paralist} % very flexible & customisable lists (eg. enumerate/itemize, etc.)
\usepackage{verbatim} % adds environment for commenting out blocks of text & for better verbatim
\usepackage{subfig} % make it possible to include more than one captioned figure/table in a single float
\usepackage{fancyhdr} % This should be set AFTER setting up the page geometry
\pagestyle{fancy} % options: empty , plain , fancy
\renewcommand{\headrulewidth}{0pt} % customise the layout...
\lhead{}\chead{}\rhead{}
\lfoot{}\cfoot{\thepage}\rfoot{}
\usepackage{sectsty}
\allsectionsfont{\sffamily\mdseries\upshape} % (See the fntguide.pdf for font help)
\usepackage[nottoc,notlof,notlot]{tocbibind} % Put the bibliography in the ToC
\usepackage[titles,subfigure]{tocloft} % Alter the style of the Table of Contents
\renewcommand{\cftsecfont}{\rmfamily\mdseries\upshape}
\renewcommand{\cftsecpagefont}{\rmfamily\mdseries\upshape} % No bold!
\usepackage{verbatim}
\usepackage{amsmath}
\usepackage{amssymb}

\DeclareMathSizes{10}{10}{10}{10}

%
% The probability of seeing a whole genome
%

\begin{document}


\section{Mike Questions}
\begin{enumerate}
\item What is the difference between $\vec{c_i}$ and $c_i$? Should we just use $\vec{c}$ and $c_i$?
\item Why does a $\vec{c_i}$ term appear in the denominator? is $\omega_j$ a function of $\vec{c_i}$ in this notation?
\item $\vec{c}$ denotes the entire genome, correct?
\item If $\vec{c}$ does denote the whole genome, should $\phi$ have some sort of subscript relating it to the current gene in the genome?
\item Is it alright if I use $\nu$ to denote the number of synonyms to $c_i$?
\item Where do the mutation terms go? I think I'll include them for completeness. Just turn
$\mbox{exp}\left[-\sum_j \beta_j \omega_j (\vec{c}_\iota) \phi \right]$
into
$\mbox{exp}\left[-\sum_j (\mu_j - \beta_j \omega_j (\vec{c}_\iota)) \phi \right]$?
\item $\mathbb{C}$ only contains synonymous sequences, correct?
\end{enumerate}


\section{Notation}

$\vec{c}$ is the vector of codons (of length $n$, the length of the ??genome/gene??)

??$\vec{c}_\iota$/$c_\iota$?? is the $i^{th}$ codon of $\vec{c}$

$\vec{\theta}$ is our given conditions, including $\beta, \phi, \omega$, and the effective population size

$\sum_j$ is shorthand for $\sum_{j=1}^n$, the sum over the whole ??genome/gene??

$\beta_j$ is the cost of codon translation. In the NSE model, this is constant. In the ROC model, this is the main variable.

$\omega_j$ is the odds of a nonsense error at codon $j$, $\frac{p_j}{1-p_j}$ 

$\phi$ is the expression level of the gene

$\mathbb{C}$ is the set of all possible ??synonymous?? sequences

$\nu$ is the number of synonymous codons to $\vec{c}_i$

\section{Math}

``The probability of seeing codon $\vec{c}_i$ at position $i$ (given our conditions) is approximately the fitness of that sequence divided by the sum of the fitness of all possible synonymous sequences"
\[
P(\vec{c}_\iota | \vec{\theta}) \approx
\frac{
\mbox{exp}\left[-\sum_{j} \beta_j \omega_j (\vec{c}_\iota) \phi \right]
}{
\sum _{K \in \mathbb{C}}
\mbox{exp}\left[-\sum_j \beta_j \omega_j (\vec{c}_\iota) \phi \right]
}
\]


%\noindent{For simplicity, I'm going to rewrite exp$\left[-\sum_{j} \beta_j \omega_j (\vec{c}_\iota) \phi \right]$ as exp[$fitness$]
Rewrite the sum as the sum across the synonyms.
Added notation: 

$\nu$ is the number of synonyms for the given codon $i$

\[
\approx
\frac{
\sum _{K \in \mathbb{C} | c_k = c_\iota}
\mbox{exp}\left[-\sum_{j} \beta_j \omega_j (\vec{c}_\iota) \phi \right]
}{
\sum_{\ell=1}^\nu
\sum _{K \in \mathbb{C} | c_k = c_\ell}
\mbox{exp}\left[-\sum_j \beta_j \omega_j (\vec{c}_\iota) \phi \right]
}
\]

Pull out the term for the current codon chain, where $c_j=c_i$

\[
\approx
\frac{
\left[
\sum _{K \in \mathbb{C}}
\mbox{exp}\left[-\sum_{j\neq k} \beta_j \omega_j (\vec{c}_\iota) \phi \right]
\right]
\left(\mbox{exp}\left[-\beta_k \omega_k (\vec{c}_\iota) \phi \right]\right)
}{
\sum_{\ell=1}^\nu
\sum _{K \in \mathbb{C} | c_k = c_\ell}
\mbox{exp}\left[-\sum_j \beta_j \omega_j (\vec{c}_\iota) \phi \right]
}
\]


Which cancels to


\[
\approx
\frac{
\mbox{exp}\left[-\beta_k \omega_k (\vec{c}_\iota) \phi \right]
}{
\sum_{\ell=1}^\nu
\mbox{exp}\left[-\sum_j \beta_j \omega_j (\vec{c}_\iota) \phi \right]
}
\]

\end{document}
