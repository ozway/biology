% !TEX TS-program = pdflatex
% !TEX encoding = UTF-8 Unicode

% This is a simple template for a LaTeX document using the "article" class.
% See "book", "report", "letter" for other types of document.

\documentclass[11pt]{article} % use larger type; default would be 10pt


\usepackage{ulem}
\newcommand\NoIndent[1]{%
  \par\vbox{\parbox[t]{\linewidth}{#1}}%
}


\usepackage[utf8]{inputenc} % set input encoding (not needed with XeLaTeX)

%%% Examples of Article customizations
% These packages are optional, depending whether you want the features they provide.
% See the LaTeX Companion or other references for full information.

%%% PAGE DIMENSIONS
\usepackage{geometry} % to change the page dimensions
\geometry{a4paper} % or letterpaper (US) or a5paper or....
% \geometry{margin=2in} % for example, change the margins to 2 inches all round
% \geometry{landscape} % set up the page for landscape
%   read geometry.pdf for detailed page layout information

\usepackage{graphicx} % support the \includegraphics command and options

% \usepackage[parfill]{parskip} % Activate to begin paragraphs with an empty line rather than an indent

%%% PACKAGES
\usepackage{booktabs} % for much better looking tables
\usepackage{array} % for better arrays (eg matrices) in maths
\usepackage{paralist} % very flexible & customisable lists (eg. enumerate/itemize, etc.)
\usepackage{verbatim} % adds environment for commenting out blocks of text & for better verbatim
\usepackage{subfig} % make it possible to include more than one captioned figure/table in a single float
% These packages are all incorporated in the memoir class to one degree or another...

%%% HEADERS & FOOTERS
\usepackage{fancyhdr} % This should be set AFTER setting up the page geometry
\pagestyle{fancy} % options: empty , plain , fancy
\renewcommand{\headrulewidth}{0pt} % customise the layout...
\lhead{}\chead{}\rhead{}
\lfoot{}\cfoot{\thepage}\rfoot{}

%%% SECTION TITLE APPEARANCE
\usepackage{sectsty}
\allsectionsfont{\sffamily\mdseries\upshape} % (See the fntguide.pdf for font help)
% (This matches ConTeXt defaults)

%%% ToC (table of contents) APPEARANCE
\usepackage[nottoc,notlof,notlot]{tocbibind} % Put the bibliography in the ToC
\usepackage[titles,subfigure]{tocloft} % Alter the style of the Table of Contents
\renewcommand{\cftsecfont}{\rmfamily\mdseries\upshape}
\renewcommand{\cftsecpagefont}{\rmfamily\mdseries\upshape} % No bold!

%%% END Article customizations


\usepackage{verbatim}
\usepackage{amsmath}






\title{Work Log for August}
\author{Logan Brown}
%\date{} % Activate to display a given date or no date (if empty),
         % otherwise the current date is printed 

\begin{document}
%\maketitle
%\tableofcontents
%\newpage


%%%NEW DAY%%%
\setcounter{section}{10}
\setcounter{subsection}{-1}
\setcounter{subsubsection}{0}

\section{August 11th}
\subsection{Goals}
\begin{enumerate}

\item Install \LaTeX
\item Local CUBFits Installation
\item Preliminary Understanding of
\begin{itemize}

\item Sella and Hirch paper~~~- link biology and statistical mechanics concepts
\item GenomeGroup paper~~~- fit the model knowing the expression levels
\item Wallace et al. paper?~~~- MCMC stuff
\item Murray et al. paper?~~~- framework

\end{itemize}
\end{enumerate}

\subsection{Progress/Notes}

\subsubsection{\LaTeX~installation}

Began a \LaTeX~installation on Tremont, installed to the home directory. It should be usable on every computer.

I ran a full installation, which may have been a mistake. It took at least 7 hours, it was still installing when I left.

\subsubsection{Local CUBFits Installation}

I locally installed CUBFits, as well as the prerequisites (seqinr, VGAM, doSNOW,  coda, EMCluster) to my home directory.

\begin{enumerate}
\item R
\item .libPaths(``{\raise.17ex\hbox{$\scriptstyle\sim$}}/cubfitsLocal"}~~~~\#add local installation to library path)
\item library(cubfits)~~~~\#load the cubfits library
\item demo(roc.train, 'cubfits')
\end{enumerate}

Doing so fixes the problem of "acceptance not in range", but introduces a new problem, "iterations terminated because half-step sizes are very small". It seems that cubfits-master is installed on Gauley, but cubfits 0.1 fixes some of those problems.

\verbatiminput{gauleyerror.txt}

\subsubsection{Readings}
First read of the Stella and Hirch paper. It helped develop my intuitions regarding CUBFits/SEMPRR, though it seems to be only tangentially related. It's good for drawing these connections, which will help with understanding and analyzing the code (indirectly)

\subsection{Future Goals}
\begin{enumerate}
\item Test \LaTeX~installation
\item Continue Readings
\item Continue analysing Gauley errors?
\end{enumerate}
%%%END OF DAY%%%

\end{document}
