% !TEX TS-program = pdflatex
% !TEX encoding = UTF-8 Unicode

% This is a simple template for a LaTeX document using the "article" class.
% See "book", "report", "letter" for other types of document.

\documentclass[11pt]{article} % use larger type; default would be 10pt


\usepackage{ulem}
\newcommand\NoIndent[1]{%
  \par\vbox{\parbox[t]{\linewidth}{#1}}%
}


\usepackage[utf8]{inputenc} % set input encoding (not needed with XeLaTeX)

%%% Examples of Article customizations
% These packages are optional, depending whether you want the features they provide.
% See the LaTeX Companion or other references for full information.

%%% PAGE DIMENSIONS
\usepackage{geometry} % to change the page dimensions
\geometry{a4paper} % or letterpaper (US) or a5paper or....
% \geometry{margin=2in} % for example, change the margins to 2 inches all round
% \geometry{landscape} % set up the page for landscape
%   read geometry.pdf for detailed page layout information

\usepackage{graphicx} % support the \includegraphics command and options

% \usepackage[parfill]{parskip} % Activate to begin paragraphs with an empty line rather than an indent

%%% PACKAGES
\usepackage{booktabs} % for much better looking tables
\usepackage{array} % for better arrays (eg matrices) in maths
\usepackage{paralist} % very flexible & customisable lists (eg. enumerate/itemize, etc.)
\usepackage{verbatim} % adds environment for commenting out blocks of text & for better verbatim
\usepackage{subfig} % make it possible to include more than one captioned figure/table in a single float
% These packages are all incorporated in the memoir class to one degree or another...

%%% HEADERS & FOOTERS
\usepackage{fancyhdr} % This should be set AFTER setting up the page geometry
\pagestyle{fancy} % options: empty , plain , fancy
\renewcommand{\headrulewidth}{0pt} % customise the layout...
\lhead{}\chead{}\rhead{}
\lfoot{}\cfoot{\thepage}\rfoot{}

%%% SECTION TITLE APPEARANCE
\usepackage{sectsty}
\allsectionsfont{\sffamily\mdseries\upshape} % (See the fntguide.pdf for font help)
% (This matches ConTeXt defaults)

%%% ToC (table of contents) APPEARANCE
\usepackage[nottoc,notlof,notlot]{tocbibind} % Put the bibliography in the ToC
\usepackage[titles,subfigure]{tocloft} % Alter the style of the Table of Contents
\renewcommand{\cftsecfont}{\rmfamily\mdseries\upshape}
\renewcommand{\cftsecpagefont}{\rmfamily\mdseries\upshape} % No bold!

%%% END Article customizations


\usepackage{verbatim}
\usepackage{amsmath}






\title{Work Log for August}
\author{Logan Brown}
%\date{} % Activate to display a given date or no date (if empty),
         % otherwise the current date is printed 

\begin{document}
%\maketitle
%\tableofcontents
%\newpage


\newpage
\setcounter{section}{19} %todays date minus 1
\setcounter{subsection}{-1}
\setcounter{subsubsection}{0}

\section{August 20th}
\subsection{Goals for the day}
\textit{Goals from Last Time}
\begin{enumerate}
\item \textit{Continue analyzing my.cubappr.r}
\item \textit{Look for Wei-Chen code on the NSE model}
\item \textit{Look at REU results (esp. a1 discussion?)}
\item \textit{Look into workflow programs for R}
\item \textit{Read about Data Structures in R user manual}

~

\NoIndent{Additional Goals}
\item Automate parts of the .tex process


\end{enumerate}

\subsection{Progress/Notes}
\subsubsection{Continue analyzing my.cubappr.r}
Wei-Chen included a breakdown of all the arguments to the cubappr.r in cubfits.pdf page 14.
He also explains the data formats on page 19.

b is apparently... 
"A named list A contains amino acids. Each element of the list A[[i]] is a list of elements
coefficients (coefficients of log(mu) and Delta.t), coef.mat (matrix format of coefficients),
and R (covariance matrix of coefficients). Note that coefficients and R are typically as
in the output of vglm() of VGAM package. Also, coef.mat and R may miss in some cases.
e.g. A[[i]]\$coef.mat is the regression beta matrix of i-th amino acid."

What is logL? Log likelihood


Here's the values for the variables, instead of using that confusing .CF.CONF\$ values

\begin{tabular}{rll}
reu13.df.obs & = & some huge matrix that looks like codon usage levels.\\
%cubfits.pdf says it also has synonyms. \\
phi.pred.Init & = & matrix containing phi guesses \\
y & = & a vector that has codon counts \\
nIter & = & 20 \\
b.Init & = & NULL \\
init.b.Scale & = & 1 \\
b.drawScale & = & 1 \\
b.RInit & = & NULL \\
p.Init & = & NULL \\
p.nclass & = & 2 \\
p.DrawScale & = & 0.1 \\
phi.pred.DrawScale & = & 1 \\
model & = & ``roc" \\
adaptive & = & ``logonormal" \\
verbose & = & TRUE \\
iterThin & = & 1 \\
report & = & 5
\end{tabular}

It's notable that the code frequently calls things as ``variable[1]", instead of just calling "variable", even if the variable only has one item in it. Ex. model[1] is totally unnecessary, model is just ``roc".

~

b.Mat is a 10 by 21 matrix (nBparams by nSave or  (\# of regression parameters) by (space for iterations)). Looks like it will contain log(mu) and Delta.t

p.Mat is a 2 by 21 matrix (nPrior (\# of prior params) by nSave)

phi.pred.Mat is a 100 by 21 matrix (n.G (number of Genes) by nSave )

logL.Mat is a 1 by 21 matrix (1 by nSave)

~

Mostly everything before the MCMC is just initializing variables, moving things around in memory. 

\textbf{my.DrawBConditionalAll is my.drawBConditionalAll.RW\_NORM}

~

I got to line 161 (including my comments 158 without them): "p.Curr $<$- .cubfitsEnv\$my.pPropTypeNoObs(n.G, phi.Curr, p.Curr, hp.param)"


\subsubsection{Look for Wei-Chen code on the NSE model}
I found the code in the CRAN cubfits installation, in demo. It's nearly identical to the roc.appr demo, but model is set to "nsef" instead of "roc".

\subsubsection{Look at REU results}

\subsubsection{Look into workflow programs for R}
So far, I've replaced this with R's build in Debug program.

\begin{enumerate}
\item library(cubfits)
\item debug(cubappr)
\item demo(roc.appr, 'cubfits')
\end{enumerate}

\subsubsection{Read about Data Structures in R user manual}

\subsubsection{Automate parts of the .tex process}
wrote writeGoals.cpp, which should save me an immense amount of time spent formatting.

\subsection{Future Goals}
\begin{enumerate}
\item Finish analyzing cubappr.r (especially the MCMC after line 158)
\item Look at the consequences if model is "nsef" instead of "roc"
\item REU Results
\item Data Structures in the R manual (list?)
\end{enumerate}


\end{document} %End of day document, REMOVE
