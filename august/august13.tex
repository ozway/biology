% !TEX TS-program = pdflatex
% !TEX encoding = UTF-8 Unicode

% This is a simple template for a LaTeX document using the "article" class.
% See "book", "report", "letter" for other types of document.

\documentclass[11pt]{article} % use larger type; default would be 10pt


\usepackage{ulem}
\newcommand\NoIndent[1]{%
  \par\vbox{\parbox[t]{\linewidth}{#1}}%
}


\usepackage[utf8]{inputenc} % set input encoding (not needed with XeLaTeX)

%%% Examples of Article customizations
% These packages are optional, depending whether you want the features they provide.
% See the LaTeX Companion or other references for full information.

%%% PAGE DIMENSIONS
\usepackage{geometry} % to change the page dimensions
\geometry{a4paper} % or letterpaper (US) or a5paper or....
% \geometry{margin=2in} % for example, change the margins to 2 inches all round
% \geometry{landscape} % set up the page for landscape
%   read geometry.pdf for detailed page layout information

\usepackage{graphicx} % support the \includegraphics command and options

% \usepackage[parfill]{parskip} % Activate to begin paragraphs with an empty line rather than an indent

%%% PACKAGES
\usepackage{booktabs} % for much better looking tables
\usepackage{array} % for better arrays (eg matrices) in maths
\usepackage{paralist} % very flexible & customisable lists (eg. enumerate/itemize, etc.)
\usepackage{verbatim} % adds environment for commenting out blocks of text & for better verbatim
\usepackage{subfig} % make it possible to include more than one captioned figure/table in a single float
% These packages are all incorporated in the memoir class to one degree or another...

%%% HEADERS & FOOTERS
\usepackage{fancyhdr} % This should be set AFTER setting up the page geometry
\pagestyle{fancy} % options: empty , plain , fancy
\renewcommand{\headrulewidth}{0pt} % customise the layout...
\lhead{}\chead{}\rhead{}
\lfoot{}\cfoot{\thepage}\rfoot{}

%%% SECTION TITLE APPEARANCE
\usepackage{sectsty}
\allsectionsfont{\sffamily\mdseries\upshape} % (See the fntguide.pdf for font help)
% (This matches ConTeXt defaults)

%%% ToC (table of contents) APPEARANCE
\usepackage[nottoc,notlof,notlot]{tocbibind} % Put the bibliography in the ToC
\usepackage[titles,subfigure]{tocloft} % Alter the style of the Table of Contents
\renewcommand{\cftsecfont}{\rmfamily\mdseries\upshape}
\renewcommand{\cftsecpagefont}{\rmfamily\mdseries\upshape} % No bold!

%%% END Article customizations


\usepackage{verbatim}
\usepackage{amsmath}






\title{Work Log for August}
\author{Logan Brown}
%\date{} % Activate to display a given date or no date (if empty),
         % otherwise the current date is printed 

\begin{document}
%\maketitle
%\tableofcontents
%\newpage


\setcounter{section}{12}
\setcounter{subsection}{-1}
\setcounter{subsubsection}{0}

\section{August 13th}
\subsection{Goals for the day}

\textit{Goals from Last Time}
\begin{enumerate}
\item \textit{Test \LaTeX~installation}
\item \textit{Continue Readings}
\item \textit{Continue analysing Gauley errors?}

~

\NoIndent{Additional Goals}

\item Look at the code
\item Look into potential Tremont errors?
\item Talk to Cedric about errors
\end{enumerate}

\subsection{Progress/Notes}
\subsubsection{Test \LaTeX~installation}
\LaTeX~installation works?

Compiles, works just fine. Was a slight adjustment from the windows MikTex GUI, but largely the same.
Developed the format used here.

\subsubsection{Readings}

\subsubsection{Gauley Errors}

Cedric says that the Gauley errors AREN'T ERRORS. It's actually just a regular part of the code output. Also the ``\#warning:iterations terminated because half-step sizes are very small" isn't a problem. Copasetic.

\subsubsection{Code Examination}
It seems like I should examing roc.appr.r first. any *appr* is for approximating the codon bias without knowing the expression levels, which is my goal.

What does roc.appr.r do?
\begin{enumerate}	%roc.appr.r enumeration
\item Get initial $\phi$ value from data/ex.test.rda
\item Change renew.iter variable from 100 to 3 in .CF.AC, which is in data/control.r
\item Calls cubappr to do the actual approximations. Times this process. cubappr is exported to mycubappr in mycubappr.r
	\begin{enumerate}	%cubappr enumeration
	\item Setup model and check data, including data structures and storage
	\item Initialize $\phi$ using my.pInit in my.init\_param.r. Uses EMCluster iff estimate.bias.Phi (false by default)
	\item Initialize $\beta$, a vector of amino acids
	\item Runs MCMC, starting at line 124
		\begin{enumerate}	%MCMC enumeration
		\item my.drawBConditionalAll 
Updates $\beta$. Seems quite important. uses VGAM? Further study
		\item Other parameters:  nu.Phi and sigma.Phi
		\item 
		\end{enumerate}		%end MCMC enumeration


	\end{enumerate} 	%end cubappr enumeration
\end{enumerate}		%end roc.appr enumeration

\subsubsection{Tremont Errors?}
Likely because I was using the cubfits-0.1.0, the version on CRAN, while I should be using cubfits-0.1.1, the version in cubfits-master.tar.gz
Moving to Gauley?

\subsection{Future Goals}
\begin{enumerate}
\item Continue analyzing cubfits-master/R files, especially mycubappr.r 
\begin{enumerate}
\item Especially study the MCMC from mycubappr.r 124-196
\item nu.Phi and sigma.Phi?
\end{enumerate}

\item Read REU Paper on Nonsense Model
\end{enumerate}

\end{document} %End of day document, REMOVE
