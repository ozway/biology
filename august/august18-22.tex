% !TEX TS-program = pdflatex
% !TEX encoding = UTF-8 Unicode

% This is a simple template for a LaTeX document using the "article" class.
% See "book", "report", "letter" for other types of document.

\documentclass[11pt]{article} % use larger type; default would be 10pt


\usepackage{ulem}
\newcommand\NoIndent[1]{%
  \par\vbox{\parbox[t]{\linewidth}{#1}}%
}


\usepackage[utf8]{inputenc} % set input encoding (not needed with XeLaTeX)

%%% Examples of Article customizations
% These packages are optional, depending whether you want the features they provide.
% See the LaTeX Companion or other references for full information.

%%% PAGE DIMENSIONS
\usepackage{geometry} % to change the page dimensions
\geometry{a4paper} % or letterpaper (US) or a5paper or....
% \geometry{margin=2in} % for example, change the margins to 2 inches all round
% \geometry{landscape} % set up the page for landscape
%   read geometry.pdf for detailed page layout information

\usepackage{graphicx} % support the \includegraphics command and options

% \usepackage[parfill]{parskip} % Activate to begin paragraphs with an empty line rather than an indent

%%% PACKAGES
\usepackage{booktabs} % for much better looking tables
\usepackage{array} % for better arrays (eg matrices) in maths
\usepackage{paralist} % very flexible & customisable lists (eg. enumerate/itemize, etc.)
\usepackage{verbatim} % adds environment for commenting out blocks of text & for better verbatim
\usepackage{subfig} % make it possible to include more than one captioned figure/table in a single float
% These packages are all incorporated in the memoir class to one degree or another...

%%% HEADERS & FOOTERS
\usepackage{fancyhdr} % This should be set AFTER setting up the page geometry
\pagestyle{fancy} % options: empty , plain , fancy
\renewcommand{\headrulewidth}{0pt} % customise the layout...
\lhead{}\chead{}\rhead{}
\lfoot{}\cfoot{\thepage}\rfoot{}

%%% SECTION TITLE APPEARANCE
\usepackage{sectsty}
\allsectionsfont{\sffamily\mdseries\upshape} % (See the fntguide.pdf for font help)
% (This matches ConTeXt defaults)

%%% ToC (table of contents) APPEARANCE
\usepackage[nottoc,notlof,notlot]{tocbibind} % Put the bibliography in the ToC
\usepackage[titles,subfigure]{tocloft} % Alter the style of the Table of Contents
\renewcommand{\cftsecfont}{\rmfamily\mdseries\upshape}
\renewcommand{\cftsecpagefont}{\rmfamily\mdseries\upshape} % No bold!

%%% END Article customizations


\usepackage{verbatim}
\usepackage{amsmath}






\title{Work Log for August}
\author{Logan Brown}
%\date{} % Activate to display a given date or no date (if empty),
         % otherwise the current date is printed 

\begin{document}
\maketitle
%\tableofcontents


\setcounter{section}{02} %week number minus 1
\setcounter{subsection}{-1}
\setcounter{subsubsection}{0}

\section{Week of August 18th-22nd}
\subsection{Goals for the Week}
\subsubsection{Set up GitHub}
Username: ozway
I've forked cubfits and added a worklog directory.
\begin{enumerate}
\item cd ~/worklog
\item git commit -a
\item git push
\end{enumerate}

\subsubsection{Read Gilchrist 2014}
See full notes in august18.tex or august18.pdf

Does this contain any information about the code structure? I don't see any description of what B is, or it's purposes. Unless B is $\beta$?

\subsubsection{Various Latex changes}
I've added a weekly summary, and automated some of the process.

\subsubsection{Look at REU Results}


\subsubsection{Workflow Tracker (like Doxygen) for R?}
\subsubsection{Study data structures in R user Manual}
list() may be interesting.

\subsubsection{study my.cubappr.r}
Broken down all the values of the variables (for the test case), which was illuminating. The values were obscured by initialization functions that were hard to access. Full details are in august22.tex or august22.pdf

I'm now breaking down the workings of the MCMC, the most important part of the code (and the best opening for parallelization)

It's notable that the differences between the (possibly defunct) NSE demo and the working ROC demo is just changing the value of model [1].

\subsection{Goals for next Week}
\begin{enumerate}
\item Finish analyzing cubappr.r (especially the MCMC after line 158)
\item Look at the consequences if model is "nsef" instead of "roc"
\item REU Results
\item Data Structures in the R manual (list?)
\end{enumerate}


\end{document} %End of day document, REMOVE
